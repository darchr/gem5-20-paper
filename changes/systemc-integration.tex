\subsection[SystemC Integration]{SystemC Integration}

While the open and configurable architecture of gem5 is of particular interest
in academia, the industry's main tool for virtual prototyping is SystemC
Transaction Level Modelling (TLM)~\cite{systemc_ieee11}. Many hardware vendors
provide SystemC TLM models of their IP and there are tools, such as Synopsys
Platform Architect\footnote{\url{https://www.synopsys.com/verification/virtual-prototyping/platform-architect.html}},
that assist in building a virtual system and analyzing it. Also, many research
projects use SystemC TLM, as they benefit from the rich ecosystem of accurate
off-the-shelf models of real hardware components. However, there is a lack of
accurate and modifiable CPU models in SystemC since the model providers want to
protect their IP. This makes the combination of gem5 with SystemC very
attractive.

\subsubsection[gem5 to SystemC Bridge]{gem5 to SystemC Bridge\footnote{By Chistian Menard, Matthias Jung, Abdul Mutaal Ahmad, and Jeronimo Castrillon}}

SystemC TLM and gem5 were developed around the same time and are based on
similar underlying ideas. As a consequence, the hardware model used by TLM is
surprisingly close to the model of gem5. In both approaches, the system is
organized as a set of components that communicate by exchanging data packets
via a well defined protocol. The protocol abstracts over the physical
connection wires that would be used in a register transfer level (RTL)
simulation and thereby significantly increases simulation speed. In gem5,
components use \emph{master} and \emph{slave} ports to communicate to other
components, whereas in SystemC TLM, connections are established via
\emph{initiator} and \emph{target} sockets. Also, the three protocols
\emph{atomic}, \emph{timing} and \emph{functional} provided by gem5 find their
equivalent in the \emph{blocking}, \emph{non-blocking} and \emph{debug}
protocols of TLM. The major difference in both protocols is the treatment of
backpressure, which is implemented by a retry phase in gem5 and with the
exclusion rule of TLM.

\begin{figure}
    \centering
    \subcaptionbox{gem5 to SystemC\label{fig:example:gem5_to_sc}}{
      \includegraphics[height=4cm]{fig/gem5_to_systemc.pdf}
    }
    \subcaptionbox{SystemC to gem5\label{fig:example:sc_to_gem5}}{
      \includegraphics[height=4cm]{fig/systemc_to_gem5.pdf}
    }
    \subcaptionbox{both directions\label{fig:example:twoway}}{
      \includegraphics[height=4cm]{fig/twoway.pdf}
    }
    \caption{Possible scenarios for binding gem5 and SystemC.}
    \label{fig:gem5_tlm_example}
\end{figure}

The similarity of the two approaches enabled us to create a light-weight
compatibility layer. In our approach, co-simulation is achieved by hosting the
gem5 simulation on top of a SystemC simulation. For this, we replaced the gem5
discrete event kernel with a SystemC process that is managed by the SystemC
kernel. A set of transactors further enables communication between the two
simulation domains by translating between the two protocols as is shown in
Figure~\ref{fig:gem5_tlm_example}. This work was published
in~\cite{menard2017-system-systemc} where we documented our approach and showed
that the transaction between gem5 and TLM only introduces a low overhead of
about \(8\%\). The source code as well as basic usage examples can be found in
\texttt{util/tlm} of the gem5 repository.

\subsubsection[SystemC in gem5]{SystemC in gem5\footnote{By Gabriel Black}}

Haven't heard anything back, yet.
