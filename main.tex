\documentclass[sigconf,nonacm,screen=true]{acmart}

\usepackage{listings}
\usepackage{subcaption}

% Separate multiple footnotes with comas
\let\oldFootnote\footnote
\newcommand\nextToken\relax

\renewcommand\footnote[1]{%
    \oldFootnote{#1}\futurelet\nextToken\isFootnote}

\newcommand\isFootnote{%
    \ifx\footnote\nextToken\textsuperscript{,}\fi}

\begin{document}
% Title portion
\title{The gem5 Simulator: Version 20.0+}
\subtitle{A new era for the open-source computer architecture simulator}
\author{Jason Lowe-Power}
\orcid{0000-0002-8880-8703}
\affiliation{%
  \institution{University of California, Davis}
}
\email{jlowepower@ucdavis.edu}

\author{Bobby R. Bruce}
\orcid{0000-0001-6070-9722}
\affiliation{%
  \institution{University of California, Davis}
}
\email{bbruce@ucdavis.edu}

\author{Many many others}

% Matthias Jung (matthias.jung@iese.fraunhofer.de) [DRAMPower and Powerdown and Elastic Traces]
% Subash Kannoth (subash.kannoth@iese.fraunhofer.de) [DRAMPower]
% Omar Naji (naji_omar@hotmail.com) [DRAMPower and Powerdown]
% Éder F. Zulian (zulian@eit.uni-kl.de) [HMC Model and DRAMPower]
% Christian Weis (weis@eit.uni-kl.de) [DRAMPower and Powerdown and Elastic Traces]
% Norbert Wehn (wehn@eit.uni-kl.de) [DRAMPower and Powerdown and Elastic Traces and SystemC]
% Abdul Mutaal Ahmad (abdul.mutaal@gmail.com) [HMC Model and SystemC]
% Thomas Grass (?) [Elastic Traces]
% Andreas Hansson (?) [Elastic Traces and DRAMPower and Powerdown]
% Stephan Diestelhorst (stephan.diestelhorst@gmail.com) [Elastic Traces]
% Wendy Elsasser (wendy.elsasser@arm.com) [DRAMPower and Powerdown and Elastic Traces]
% Radhika Jagtap (radhika.jagtap@arm.com) [Powerdown and Elastic Traces]


\titlenote{
    gem5 is the result of the merger of the GEMS project started in 1999, and the m5 project started in 2003.
    Development of gem5 has been active for about 20 years, and this version is being published in 2020. Thus, ``gem5-20''.
}

\begin{abstract}
    The open-source and community-supported gem5 simulator is one of the most popular tools for computer architecture research.
    This simulation infrastructure allows researchers to model modern computer hardware at the cycle level, and it has enough fidelity to boot unmodified Linux-based operating systems and run full applications for multiple architectures including x86, Arm\textregistered, and RISC-V.
    The gem5 simulator has been under active development over the last nine years since the original gem5 release.
    In this time, there have been over 7500 commits to the codebase from over 250 unique contributors which have improved the simulator by adding new features, fixing bugs, and increasing the code quality.
    In this paper, we give an overview of gem5's usage and features, describe the current state of the gem5 simulator, and enumerate the major changes since the initial release of gem5.
    We also discuss how the gem5 simulator has transitioned to a formal governance model to enable continued improvement and community support for the next 20 years of computer architecture research.
\end{abstract}

\maketitle
\renewcommand{\shortauthors}{Lowe-Power and the gem5 Community}

This is a placeholder for the introduction text.

Dummy citation. Having at least one citation keeps bibtex
happy~\cite{Binkert-gem5-2011}.

\section{Major changes in gem5-20}
\label{sec:changes}

In addition to the systematic changes in project management discussed in Section~\ref{sec:current-gem5} there have also been many added features, fixed bugs, and general improvements to the codebase.
This section contains descriptions of some of the major changes to gem5.
There were 7015 commits between when gem5 was released (in 2011) and the release of gem5-20 by at least 250 unique contributors.
This section is a comprehensive, but not exhaustive, list of the major changes in gem5.
Along with the description of the changes in gem5, we also recognize the individuals or groups who made significant contributions to each of these features with separate by-lines for each subsection.
However, there are many unlisted contributors that were indispensable in getting gem5 where it is today.

Table~\ref{table:overview} gives and overview of the major changes in gem5 with pointers to subsections which contain more detail on each change.

\begin{table}
    \begin{tabular}{|p{0.95\linewidth}|}
        \hline
        \multicolumn{1}{|c|}{\textbf{General usability improvements} } \\
        \hline
        \textbf{Section~\ref{sec:resources}:} Added new resources repository with disk images, kernel images, etc. \\
        \textbf{Section~\ref{sec:learning}:} Learning gem5 book and class. Provides a way to get started using and developing with gem5. \\
        \hline \hline
        \multicolumn{1}{|c|}{ \textbf{ISA improvements} } \\
        \hline
        \textbf{Section~\ref{sec:riscv}:} RISC-V ISA added. User mode fully supported. Some support for full system. \\
        \textbf{Section~\ref{sec:arm}:} ARM ISA improvements. Added support for Armv8, SVE instructions, and trusted firmware. \\
        \textbf{Section~\ref{sec:x86}:} x86 ISA improvements. Better support for out-of-order CPU models, many instructions added, and support for TSO memory consistency. \\
        \hline \hline
        \multicolumn{1}{|c|}{\textbf{Execution model improvements} } \\
        \hline
        \textbf{Section~\ref{sec:predictor}:} New branch predictors including L-TAGE.\\
        \textbf{Section~\ref{sec:virtualized-ff}:} New CPU model based on KVM added. Uses host hardware to accelerate simulator. \\
        \textbf{Section~\ref{sec:elastic}:} Elastic trace execution added. Trace capture and playback with dynamic dependencies for fast flexible simulation. \\
        \hline \hline
        \multicolumn{1}{|c|}{\textbf{Memory system improvements} }\\
        \hline
        \textbf{Section~\ref{sec:dramcontroller}:} Configurable DRAM controller. Added support for many DRAM devices, low-power DRAM, quality of service, and power models. \\
        \textbf{Section~\ref{sec:classic}:} Classic cache improvements. Added non-coherent caches, write streaming optimizations, cache maintenance operations, and snoop filtering. \\
        \textbf{Section~\ref{sec:replacement}:} General replacement policy framework and cache compressions support added. \\
        \textbf{Section~\ref{sec:ruby}:} Ruby model improvements. Many general improvements, GPU coherence protocols, and support for ARM ISA. \\
        \textbf{Section~\ref{sec:garnet}:} Garnet network improved to version 2.0 with more detailed router and network models. \\
        \hline \hline
        \multicolumn{1}{|c|}{\textbf{New models} } \\
        \hline
        \textbf{Section~\ref{sec:gpu}:} GPU compute model added. Models AMD's GCN architecture in SE-mode with support for shared memory systems. Tests for GPU-like coherence protocols also added. \\
        \textbf{Section~\ref{sec:dvfs}:} Runtime power modeling and DVFS support added. \\
        \textbf{Section~\ref{sec:virtio-nomali}:} Support for timing-agnostic device models added. VirtIO enables more flexible guest-simulator interaction and the NoMali GPU model allows graphic-based applications to execute more realistically. \\
        \textbf{Section~\ref{sec:dist-gem5}:} Support for modeling multiple distributed systems added. \\
        \textbf{Section~\ref{sec:systemc}:} SystemC model integration. Added a bridge from SystemC TLM models to gem5 models, and added an implementation of SystemC for tight gem5-SystemC integration. \\
        \hline \hline
        \multicolumn{1}{|c|}{\textbf{General infrastructure improvements} } \\
        \hline
        \textbf{Section~\ref{sec:se-mode}:} SE-mode improvements. Support for dynamically-linked binaries, more system calls, multi-threaded applications, and a virtual file system. \\
        \textbf{Section~\ref{sec:testing}:} Testing improvements. New unit test framework and continuous integration support. \\
        \textbf{Section~\ref{sec:internal}:} General infrastructure improvements. Added support for HD5F output for statistics, Python 3 support, and asynchronous modeling. \\
        \textbf{Section~\ref{sec:guest-sim}:} Updated guest$\leftrightarrow$simulator APIs. \\
        \hline
    \end{tabular}
    \caption{Overview of major change in gem5.}
    \label{table:overview}
\end{table}

% Usability
\subsection[gem5 resources]{gem5 resources\footnote{by Ayaz Akram, Bobby R. Bruce, Hoa Nguyen, and Mahyar Samani}}
\label{sec:resources}

Though the gem5 simulator permits the simulation of many different system
using a variety of benchmarks and tests, gathering and compiling these
resources can be a laborious process. To provide
a better user-experience we have began maintaining \emph{gem5
resources}, which we broadly define as a set of artifacts
that are not required to build or run gem5, but that may be utilized to carry
out particular simulations. For example, Linux kernels, disk images, popular
benchmark suites, and commonly used tests binaries are frequently needed by
users of gem5 but are not distributed as part of the gem5 itself. As part of
our gem5-20 release, these resources, with source code and build instructions
for each, are gradually being centralized in a common repository\footnote{
\url{https://gem5.googlesource.com/public/gem5-resources}}.

A key goal of this repository is to ensure reproducibility of gem5
experiments. As a case in point, it can often be difficult to ascertain what exact
kernel configuration was used to test booting within a particular architecture
simulation. With gem5 resources, the correct kernel can be cited more easily,
thereby improving reproducibility of experiments and tests.

\subsubsection{Testing gem5-20 with gem5 resources}

Another important aim of creating a common set of gem5 resources is to more
regularly test gem5 on a suite of benchmarks and common Linux booting setups.
As part of gem5-20, we have tested the simulator's effectiveness
at running SPEC 2006~\cite{spec06}, SPEC 2017~\cite{spec17},
PARSEC~\cite{parsec}, the NAS Parallel Benchmarks (NPB)~\cite{npb},
and the GAP Benchmark Suite (GAPBS)~\cite{gapbs}. We have also shown gem5-20's
performance at running 5 different LTS Linux Kernel releases on a set of
different CPU and memory configurations. The results from these investigations
can be found on our website~\footnote{
\url{http://www.gem5.org/documentation/benchmark_status}}. We shall use this
information, and gem5 resources repository, to better target problem areas in
the gem5 project.

Furthermore, with a shared set of common resources and knowledge of what
configurations work best with gem5, we can provide the community with a set of
``known good'' gem5 configurations to facilitate computer architecture
research. We intend for these configurations to replicate the functionality and
performance of architectural components at a high level of fidelity.

%The gem5 simulator provides support for simulating many different system configurations. However,
%setting up the simulator to run simulations, specifically in full system mode, could take significant
%amount of time and become very complicated. In order to ensure the reproducibility of gem5 experiments
%many details such as code version should be documented so that created components are identical for
%reuse. gem5 resources consist of components used to conduct certain computer systems architecture
%research on known-good configurations using gem5. They include anything from the configuration
%files used to build a Linux kernel that works with gem5 to the configuration scripts that
%describe the computer system to be simulated. The provided resources have been tested with gem5-20
%and their working status and initial statistics along with their creation processes have been documented~\cite{benchmark_status}~\cite{resources-repo}.
%They could be used to do research with different system configurations and to save the user substantial amount of time.
%Moreover, some of the provided resources could be modified per user requirements such as the working Ubuntu 20.04 disk-image.
%One of the most important resources required by any full system experiment with gem5 is the disk-image
%which has one of the most time consuming and error prone build procedures. The disk-images provided by
%gem5 resources have been created by packer. We also provide gem5 resources to conduct experiments with many
%popular benchmark suites like SPEC 2006~\cite{spec06}, SPEC 2017~\cite{spec17}, PARSEC~\cite{parsec},
%NAS Parallel Benchmarks (NPB)~\cite{npb}, GAP Benhmark Suite (GAPBS)~\cite{gapbs} and Linux Kernel.

\subsection[Learning gem5]{Learning gem5\footnote{By Jason Lowe-Power}}

The gem5 simulator has a steep learning curve.
Most of the time, using gem5 in research means \emph{modifying} the simulator to change or add new models.
Not only do new users have to navigate the 100s of different models, but they also have to understand the core of the simulation framework.
We found that this steep learning curve was one of the biggest impediments to productively using gem5.
There was anecdotal evidence that it would take new users \emph{years} to learn to use gem5 effectively~\cite{Power-gem5horrors-2015}.
Additionally, the only way to learn parts of gem5 was to work with a senior graduate student or to intern at a company and pick up the knowledge ``on the job''.
Many parts of gem5 were not documented except as the source code.

\emph{Learning gem5} reduces the knowledge gap between new users and experienced gem5 developers.
Learning gem5 takes a bottom up approach to teaching new users the internals of gem5.
There are currently three parts of Learning gem5, ``Getting Started'', ``Modifying and Extending'', and ``Modeling Cache Coherence with Ruby''.
Each part walks the reader through a step-by-step coding example starting from the simplest possible design up to a more realistic example.
By explaining the thought process behind each step, the reader gets a similar experience to working alongside an experienced gem5 developer.
Learning gem5 includes documentation on the gem5 website\footnote{\url{http://www.gem5.org/documentation/learning_gem5/introduction/}} and source code in the gem5 repository for these simple ground-up models.

Looking forward, we will be significantly expanding the areas of the simulator covered by Learning gem5 and creating a gem5 ``summer school'' initially offered summer of 2020.
This ``summer school'' will mainly be an online class (e.g., Coursera) with all videos available on the gem5 YouTube channel\footnote{\url{https://www.youtube.com/channel/UCCpCGEj_835WYmbB0g96lZw}}, but we hope to have in-person versions of the class as well.
These classes will also be the basis of gem5 Tutorials held with major computer architecture and other related conferences.

% ISA improvements
\subsection[RISC-V ISA Support]{RISC-V ISA Support}
\label{sec:riscv}

RISC-V is a new ISA which has quickly gained popularity since its creation in 2010, only one year before the initial gem5 release~\cite{Waterman2011riscv}.
In this time, the number of RISC-V users has grown significantly, especially in the computer architecture research community.
Thus, the addition of RISC-V as a supported ISA for gem5 is one of the main new features in the past nine years.

\subsubsection[General RISC-V ISA Implementation]{General RISC-V ISA Implementation\footnote{By Alec Roelke}~\cite{risc5-gem5, risc5-multicore-gem5}}

The motivation for implementing the RISC-V ISA into gem5 stemmed from needing a way to explore architectural parameters for RISC-V designs.
At the time of implementation, the only means of simulating RISC-V was using Spike (its simplified, single-cycle RTL simulator), QEMU, full RTL simulation, or emulation on an FPGA.
Spike and QEMU are not detailed enough and RTL simulation is too time consuming for these methods to be feasible for architectural parameter exploration.
With FPGA emulation, it is difficult to retrieve performance information without modifying both the RTL design and the system software.
The gem5 simulator provides an easy means of performing architectural analysis through its detailed hardware models.

The implementation follows the divisions of the instruction set into its base ISA and extensions, beginning with the 32-bit integer base set, RV32I.
This implementation was modeled off of the existing gem5 code for MIPS and Alpha ISAs, which are also RISC instruction sets that share many of the same operations as RISC-V.
Including support for 64-bit addresses and data (RV64) and for the multiply (M) extension mainly involved adding the new instructions and changing some parameters to expand register and data path widths.

The next two extensions, atomic (A) and floating point (F and D for single- and double-precision, respectively), were more complicated.
The A extension includes both load-reserved/store-conditional (LR/SC) sequence of instructions for performing complex atomic operations on memory and a set of read-modify-write instructions for performing simple ones.
These instructions were implemented as a pair of micro-ops that acted like an LR/SC pair with one of the pair additionally performing the specified operation.
Floating-point instructions required many special cases to ensure correct error handling and reporting, and we were not able to implement one of the five possible rounding modes (round away from zero) RISC-V specifies for inexact calculations due to the fact that C++ does not support it.
Finally, support for the non-standard compressed (C) extension, which adds 16-bit versions of high-usage instructions, was added when it was discovered that this extension was included by default in many RISC-V software toolchains (e.g., GCC).
The compressed instruction implementation required creating a state machine in the instruction decoder to keep track of whether the current instruction is compressed, to increment the PC by the correct amount based on the size of the instruction, and to handle cases where a full-length instruction crosses a 32-bit word boundary.

With this implementation, most RISC-V Linux programs are supported in system call emulation mode.
Continued work has improved the implementation of atomic instructions, including actual atomic read-modify-write accesses in a single instruction and steps toward support for full system simulation.
Additionally, gem5's version of the RISC-V test-suite\footnote{\url{https://github.com/riscv/riscv-tests}} has been updated to the latest version and several corner cases in gem5 have been fixed, so that now most of the tests are working correctly.

\subsubsection[RISC-V Full System Support]{RISC-V Full System Support\footnote{By Nils Asmussen}}

To simulate complete operating systems the RISC-V ISA has been extended to support full system simulation.
More specifically, we added support for Sv39 paging according to the privileged ISA 1.11\footnote{\url{https://riscv.org/specifications/privileged-isa/}} with a 39-bit virtual address space, a page-table walker performing a three-level translation, and a translation lookaside buffer (TLB).
The page-table walker code is based on the existing gem5 code for x86 due to the structural similarities.
While a few steps are still missing to run Linux, general support to run a complete RISC-V operating system on gem5 is available now.

\subsection[Arm Improvements]{Arm Improvements}
\label{sec:arm}

\subsubsection[Armv8 Support]{Armv8 Support\footnote{by Giacomo Gabrielli, Javier Setoain, and Giacomo Travaglini}}

The Armv8-A architecture introduced two different architectural states:
AArch32, supporting the A32 and T32 instruction sets (backward-compatible with
Armv7-A and Thumb instruction sets, respectively), and AArch64, a new
state offering support for 64-bit addressing via the A64 instruction set. Currently, gem5
supports all of the above instruction sets and the interworking
between them.
On top of the user-level features, several important system-level extensions, e.g. the
security (aka TrustZone\textregistered~\cite{ArmTustZone}) and virtualization extensions~\cite{ArmARM}, have been contributed opening up new avenues for architectural and microarchitectural research.

While Armv8-A was a major iteration of the architecture, there have been
several smaller iterations introduced by Arm with a yearly cadence, and various
contributors have implemented some of the main features from those extensions,
up to Armv8.3-A.

\subsubsection[Support for the Arm Scalable Vector Extension (SVE)]{Support for the Arm Scalable Vector Extension (SVE)\footnote{by Giacomo Gabrielli, Javier Setoain, and Giacomo Travaglini}}

In 2016, Arm introduced their Scalable Vector Extension (SVE)~\cite{ArmARM}, a
novel approach to vector instruction sets. Instead of having fixed-size vector
registers, SVE operates on registers that can be anywhere between 128 to 2048
bit long (in 128-bit increments). SVE code is arranged in a way that is agnostic to the
underlying vector length (Vector Length Agnostic Programming), and a single SVE
instruction will perform its operation on as many elements as the vector
register can fit, depending on its length. On top of the 32 variable-length
vector registers, SVE also adds 16 variable length predicate registers for
predicated execution. These registers store one bit per byte (the minimum
element size) in the vector register, and can be used to select specific
elements in the vector for operation~\cite{white-paper-on-SVE-and-VLA-programming}.

To support SVE, gem5 implements register storage and register access
as two separated classes, a container and an interface, decoupling one from the
other. The vector registers can be of any arbitrary size and be accessed as
vectors of elements of any particular type, depending on the operand types of
each instruction. This not only facilitates handling variable size registers,
it also abstracts the nuances of handling predicate registers, where the stored
values have to be grouped and interpreted differently depending on the operand
type.

This design provides enough flexibility to support any vector instruction sets
with arbitrarily large vector registers.

\subsubsection[Trusted Firmware Support]{Trusted Firmware Support\footnote{by Adrian Herrera}}

Trusted Firmware (TF-A) is Arm's reference implementation of Secure World software for A-profile architectures.
It enables Secure Boot flow models, and provides implementations for the Secure Monitor executing at Exception Level 3 (EL3) as well as for several Arm low-level software interface standards, including System Control and Management Interface (SCMI) driver for accessing System Control Processors (SCP), Power State Coordination Interface (PSCI) library support for power management, and Secure Monitor Call (SMC) handling.

TF-A is supported on multiple Arm Development Platforms (APDs), each of them characterized by its set of hardware components and their location in the memory map (e.g., Juno ADP and the Fixed Virtual Platforms (FVP) ADP family).
However, the Arm reference platforms in gem5 are part of the \verb|VExpress_GEM5_Base| family.
These are loosely based on a Versatile\texttrademark Express RS1 platform with a slightly modified memory map. TF-A implementations are provided for both Juno and FVPs, however not for \verb|VExpress_GEM5_Base|.

Towards unifying Arm's platform landscape, we now provide a \verb|VExpress_GEM5_Foundation| platform as part of gem5's \verb|VExpress_GEM5_Base| family.
This is based on and compatible with FVP Foundation, meaning all Foundation software may run unmodified in gem5, including but not limited to TF-A.
This allows for simulating boot flows based on UEFI implementations (U-boot, EDK II), and brings us a step closer to Windows support in gem5.

\subsection[X86 ISA Improvements]{X86 ISA Improvements\footnote{by Nilay Vaish}}

The x86 or x86-64 ISA is one of the most popular ISAs for desktop, server, and high-performance compute systems.
Thus, there has been significant effort to improve gem5's modeling of this ISA.
This section presents a subset of the changes to improve the x86 ISA.
There are many other improvements large and small that generally have improved the fidelity of x86 modeling.

In out-of-order CPUs (e.g., gem5's O3CPU), instructions whose dependencies have been satisfied are allowed to execute even if there are instructions earlier in the stream waiting for their operands.
The flag register used in the x86 ISA complicates this as almost every instruction both reads and writes this register making them all dependent on one another.
Maintaining a single flag register can introduce dependencies that need not exist.
We now maintain multiple flag registers for holding subsets of flag bits to reduce the dependencies.
This prevents unnecessary serialization, unlocking a significant amount of instruction-level parallelism.

Memory consistency models decide the amount of parallelism available in a memory system, while correctly executing a program.
The x86 architecture is based on the Total Store Order (TSO) memory model~\cite{coherenceprimer}.
We added support for TSO to gem5 for the x86 architecture.
This meant ensuring that a later load from a thread can bypass earlier loads/stores, but stores from the same thread are always executed in order.
The out-of-order CPU model in gem5 has been improved to implement both TSO and more relaxed consistency models (e.g., those in the RISC-V~\ref{sec:riscv} and Arm~\ref{sec:arm} architectures).

% Execution models
% \subsection[The Minor In-Order CPU Model]{The Minor In-Order CPU Model\footnote{by Andrew Bardsley}}

Haven't heard back, yet.


\subsection[Branch Predictor Improvements]{Branch Predictor Improvements\footnote{by Dibakar Gope}}

In gem5, multiple branch prediction models are available, many of which were added since the initial release of gem5.
Currently, gem5 supports five different branch prediction techniques including the well-known TAGE predictor as well as standard predictors such as bi-mode, tournament, etc.
This list can easily be expanded to cover different variants of these well-known branch predictors.
Besides, the support for loop predictor and indirect branch predictor is also available.

Furthermore, the modularity of the implementation of different branch predictors allows ease of inclusion of secondary or side predictors into the prediction mechanism of primary predictors.
For example, TAGE can be seamlessly augmented with a loop predictor to predict loops with constant iteration numbers.
Indirect branch predictor can be made to use complex TAGE-like scheme instead of simple history-based predictors with only a few hours of development effort.
In addition to this, these different predictors can be configured with different sizes of history registers and table-like structures.
For example, TAGE predictor can be configured to run with different sizes of the history register and consequently a different number of predictor tables, allowing users to investigate the effects of different predictor sizes in various performance metrics.

Future development is planned to include the support of neural branch predictors (e.g., perceptron branch predictor, etc.) and different variants of TAGE and perceptron predictors that have demonstrated significant improvement in branch misses in recent years.

\subsection[Virtualized Fast Forwarding]{Virtualized Fast Forwarding\footnote{by Andreas Sandberg}}

I haven't heard anything back, yet.
Note: Cite paper ``Full Speed Ahead: Detailed Architectural Simulation at Near-Native Speed''

\subsection[Elastic Traces]{Elastic Traces\footnote{by Radhika Jagtap, Matthias Jung, Stephan Diestelhorst, Andreas Hansson, Thomas Grass, and Norbert Wehn}}
%
Detailed execution-driven CPU models, like gem5, offer high accuracy, but at the cost of simulation speed.
Therefore, trace-driven simulations are widely adopted to alleviate this problem, especially for studies focusing on memory-system exploration.
However, traces with fixed time stamps always include the implicit behavior of the simulated memory system with which they were recorded.
If the memory system is changed during exploration this will lead to wrong simulation results, since an out-of-order core would react differently on the new memory system.
Ideally, trace-driven core models will mimic out-of-order processors executing full-system workloads to enable computer architects to evaluate modern systems.
Therefore, we proposed the concept of elastic traces in which we accurately capture data and load/store order dependencies by instrumenting a detailed out-of-order processor model~\cite{jagdie_16}.
In contrast to existing work, we do not rely on offline analysis of timestamps, and instead use accurate dependency information tracked inside the processor pipeline.
We thereby account for the effects of speculation and branch misprediction resulting in a more accurate trace playback compared to fixed time traces.
We integrated a trace player in gem5 that honors the dependencies and thus adapts its execution time to memory-system changes, as would the actual CPU. Compared to the detailed CPU model, our trace player achieves a speed-up of 6-8 times while maintaining a high simulation accuracy (83-93\%), achieving fast and accurate system performance exploration.

% \subsection[Memory Traces and Traffic Generator]{Memory Traces and Traffic Generator\footnote{by Andreas Hanson}}

Haven't heard back, yet.

% Memory system
\subsection[Off-Chip Memory System Models]{Off-Chip Memory System Models\footnote{by Nikos Nikoleris}}

gem5 can model a large number of configurations in the off-chip memory system.
Its memory controller handles requests from the on-chip memory system and issues read and write commands for the actual memory device~\cite{}() and models their timing behavior.
Over the years a number of contributions have added features that allow modeling of emerging new technologies and features.

\subsubsection[New memory controller features]{New memory controller features\footnote{by ...}}

The gem5 DRAM controller provides the interface to external memory, which is traditionally DRAM.
The controller consists of 2 main components: the memory controller and the DRAM interface.
The DRAM interface contains media specific information, defining the architecture and timing parameters of the DRAM as well as the functions that manage the media specific operations like activation, precharge, refresh and low power modes.

\subsubsection[Low-power DDR]{Low-power DDR\footnote{by ...}}

LPDDR5 is currently in mass production for use in multiple markets including mobile, automotive, AI, and 5G.
This technology is expected to become the mainstream Flagship Low-Power DRAM by 2021 with anticipated longevity due to proposed speed grade extensions.
The specification defines a flexible architecture and multiple options to optimize across different use cases, trading off power, performance, reliability and complexity.
To evaluate these tradeoffs, we have updated the memory controller to support the new features and added LPDDR5 configurations.

While these changes have been incorporated for LPDDR5, some of them could be applicable to other memory technologies as well.
The gem5 changes incorporate new timing parameters, support of multi-cycle commands and support of interleaved bursts.
These features require new checks and optimizations in gem5 to ensure the model integrity when comparing to real hardware.
For example, support for multi-cycle commands along with the changes to LPDDR5 clocking motivated a new check in gem5 to verify command bandwidth.
Previously, the DRAM controller did not verify contention on the command bus and assumed adequate command bandwidth, but with the evolution of new technologies this assumption is not always valid.

\subsubsection[Quality of Service Extensions]{Quality of Service Extensions\footnote{by Matteo Andreozzi}}

The coexistence of heterogeneous tasks/workloads on a single computer system is common practice in modern systems, from the automotive to the high-performance computing use-case.
It allows the system to minimize costs by improving the resources utilization and improving the efficiency of data sharing across workloads.
This, however, comes at the cost of potential severe performance degradation due to interference on shared resources, and increased uncertainty in terms of workload performance predictability.

To compensate for these shortcomings, we stress the need to introduce a mechanism for predictively and deterministically managing such systems resources, i.e., providing Quality of Service (QoS).
The concept and challenges of QoS itself are not new and achieving QoS is generally hard in complex systems, as we learned from Computer Networking.
We therefore define here QoS in Systems on Chips on the following two principles:
\begin{enumerate}
    \item \emph{QoS is resource access arbitration}: a QoS-enabled resource (e.g., memory) guarantees certain Levels of Service (memory access bandwidth and latency, compute time, peripheral access, etc.) to its serviced users (e.g., software threads)
    \item \emph{QoS is quantifiable and predictable:} the set of guarantees a QoS-enabled resource can provide are known a priori and characteristic of the implemented QoS arbitration schemes.
    The level of service that a QoS enabled resource will guarantee to its users must therefore be predictable to a certain extent given a specific set of policies and their configurations.
\end{enumerate}

Quality of Service is the ability of a system to provide differential treatment to its clients, in a quantifiable and predictable way.

The contribution involved the definition of a QoS-aware memory controller in gem5, and the definition of basic (example) policies modelling the prioritization algorithm of the memory controller. Those are the Fixed priority policy (every master in the system has a fixed priority assigned) and the Proportional Fair policy (where the priority of a master is dynamically adjusted at runtime based on utilization).
The DRAM controller in gem5 had been rewritten to include the QoS changes; with the framework in place a user can write its own policy and seamlessly plug it into a real memory controller model to unlock system wide explorations under its own arbitration algorithm.

\subsubsection[DRAMPower and DRAM Power-Down Modes]{DRAMPower and DRAM Power-Down Modes\footnote{by Matthias Jung, Wendy Elsasser, Radhika Jagtap, Subash Kannoth, Omar Naji, Éder F.
Zulian, Andreas Hansson, Christian Weis, and Norbert Wehn }}
%
Across applications, DRAM is a significant contributor to the overall system power.
For example, the DRAM access energy per bit is up to three orders of magnitude higher compared to an on-chip memory access.
Therefore, an accurate and fast power estimation is crucial for an efficient design space exploration.
DRAMPower~\cite{kargoo_14} is an open source tool for fast and accurate power and energy estimation for several DRAM memories based on JEDEC standards.
It supports unique features like power-down, bank-wise power estimation, per bank refresh, partial array self-refresh, and many more.
In contrast to Micron’s DRAM Power estimation spread sheet\footnote{\url{https://www.micron.com/support/tools-and-utilities/power-calc}}, which estimates the power from device manufacturer’s data sheet and workload specifications (e.g. Rowbuffer-Hit-Rate or Read-Write-Ratio), DRAMPower uses the actual timings from the memory transactions, which leads to a much higher accuracy in power estimation.
Furthermore, the DRAMPower tool performs DRAM command trace analysis based on memory state transitions and hence, avoids cycle-by-cycle evaluation, thus speeding up simulations.

For the efficient integration of DRAMPower into gem5, we changed the tool from a standalone simulator to a library that could be used in discrete event-based simulators for calculating the power consumption online during the simulation.
Furthermore, we integrate the power-down modes into the DRAM controller model of gem5~\cite{jagjun_17} in order to provide the research community a tool for power-down analysis for a breadth of use cases. We further evaluated the model with real HPC workloads, illustrating the value of integrating low power functionality into a full system simulator.
%
\subsubsection[Future Improvements to Off Chip Memory Models]{Future Improvements to Off Chip Memory Models\footnote{by Wendy Elsasser}}

With the advent of SCM (storage class memory), emerging NVM could also exist on a memory interface, potentially alongside DRAM.
To enable support of NVM and future memory interfaces, a systematic approach was chosen to refactor the DRAM controller.
The DRAM interface was pulled out of the controller and moved to a separate DRAM interface object.
In parallel, an NVM interface was created to model an agnostic interface to emerging memory.

The DRAM interface and the NVM Interface have configurable address ranges allowing flexible heterogeneous memory configurations.
For example, single memory controller can have a DRAM interface, an NVM interface, of both interfaces defined.
Other configurations are feasible, providing a flexible framework to study new memory topologies and evaluate the placement of emerging NVM in the memory sub-system.

\subsection[Classic Caches Improvements]{Classic Caches Improvements\footnote{by Nikos Nikoleris}}
\label{sec:classic}

The classic memory system implements a snooping MOESI-like coherence protocol that allows for flexible, configurable cache hierarchies.
The coherence protocol is primarily implemented in the \lstinline|Cache| and the \lstinline|CoherentXBar| classes and the \lstinline|SnoopFilter| object implements a common optimization to reduce unnecessary coherence traffic.

Over the years, the components of the classic memory system have received significant contributions with a primary focus of adding support for future technologies and enhancing its accuracy.

\subsubsection[Non-Coherent Cache]{Non-Coherent Cache}
The cache model in gem5 implements the full coherence protocol and as a result can be used in any level of the coherent memory subsystem (e.g., as an L1 data cache or instruction cache, last-level cache, etc.).
The non-coherent cache is a stripped down version of the cache model designed to be used below the point-of-coherence (closer to memory).
Below the point-of-coherence, the non-coherent cache receives only requests for fetches and writebacks and itself send requests for fetches and writebacks to memory below.
As opposed to the regular cache, the non-coherent cache will not send any snoops to invalidate or fetch data from caches above.
As such the non-coherent cache is a greatly simplified version in terms of handling the coherence protocol compared to the regular cache while otherwise supporting the same flexibility (e.g., configurable tags, replacement policies, inclusive or exclusive, etc.).

The non-coherent cache can be used to model system-level caches, which are often larger in size and can be used by CPUs and other devices in the system.

\subsubsection[Write Streaming Optimizations]{Write Streaming Optimizations}

Write streaming is a common access pattern which is typically encountered when software initializes or copies large memory buffers (e.g., memset, memcpy).
When executed, the core issues a large number of write requests to the data cache. The data cache receives these write requests and issues requests for exclusive copies of the corresponding cache lines. To get an exclusive copy, it has to invalidate copies of that line and fetch a copy of the data (e.g., from off-chip memory). As soon as it receives data, it performs all writes for that line and often will overwrite it completely. As a result, the data cache unnecessarily fetches data only to overwrite it shortly after. Often these write buffers are large in size and also trash the data cache.

Common optimizations~\cite{10.1145/173682.165154} coalesce writes to form full cache line writes, avoid unnecessary data fetches and achieve significant reduction in memory bandwidth.
In addition, when the written memory buffer is large, we can also avoid thrashing the data cache by bypassing allocation.

We have implemented a simple mechanism to detect write streaming access patterns and enable coalescing and bypassing.
The mechanism attaches to the data cache and analyses incoming write requests. When the number of sequential writes reaches a first threshold, it enables write coalescing and when a second threshold is reached, in addition, the cache will bypass allocation for the writes in the stream.

\subsubsection[Cache Maintenance Operations]{Cache Maintenance Operations}

Typically, the contents of the cache are handled by the coherence protocol.
For most user-level code, caches are invisible.
This greatly simplifies programming and ensures software portability.
However, when interfacing with devices or persistent memory, the effect of caching becomes visible to the programmer.
In such cases, a user might have to trigger a writeback which propagates all the way to the device or the persistent memory.
In other cases, a cache invalidation will ensure that a subsequent load will fetch the newest version of the data from a buffer of the main memory.

Cache maintenance operations (CMOs) are now supported in gem5 in a way that can deal with arbitrary cache hierarchies.
An operation can either clean and/or invalidate a cache line.
A clean
operation will find the dirty copy and trigger a writeback and an invalidate operation will find all copies of the cache line and invalidate them and the combined operation will perform both actions.
The effects of CMOs are defined with reference to a configurable point in the system.
For example, a clean and invalidate sent to the point-of-coherence will find all copies of the block above the point-of-coherence, invalidate them, and if any of them is dirty also trigger a writeback to the memory below the point-of-coherence.

\subsubsection[Snooping Support and Snoop Filtering]{Snooping Support and Snoop Filtering}

In large systems, broadcasting snoop messages is slow, they cost energy and time, and they can cause significant scalability bottlenecks.
Therefore, snoop filters (also called directories) are used to keep track of which caches or nodes are keeping a copy of a particular cached line.
We added a snoop filter to gem5 which is a distributed component that keeps track of the coherence state of all lines cached ``above'' it, similar to the AMD Probe Filter~\cite{Conway:opteron:2010}.
For example, if the snoop filter sits next to the L3 cache and is accessed before the L3, it knows about all lines in the L2 and L1 caches that are connected to that L3 cache.

Using the snoop filter, we can reduce the amount of messages from $O(N^2)$ to $O(N)$ with $N$ concurrent requestors in the system.
Modeling the snoop filter separately from the cache allows us to use different organizations for the filter and the cache, and distributing area between shared caches vs coherence tracking filters.
We also model the effect of limited filter capacity through back-invalidations that remove cache entries if the filter becomes full for more realistic cache performance metrics.]s
Finally, the more centralized coherence tracking in the filter allows for better checking of correct functionality of the distributed coherence protocol in the classic memory system.

\subsection[Cache Replacement Policies and New Compression Support]{Cache Replacement Policies and New Compression Support\footnote{By Daniel Rodrigues Carvalho}}

In general, hardware components frequently contain tables, whose contents are managed by replacement policies.
In gem5, multiple replacement policies are available, which can paired with any table-like structure, allowing users to carry research on the effects of different replacement algorithms in various hardware units.
Currently, gem5 supports 13 different replacement policies including several standard policies such as LRU, FIFO, and Pseudo-LRU, and various RRIPs~\cite{Jaleel2010rrip}.
These policies can be used with both the classic caches and Ruby caches.
This list is easily expandable to cover schemes with greater complexity as well.

The simulator also supports cache compression by providing several state-of-the-art compression algorithms~\cite{sardashti2015primer} and a default compression-oriented cache organization.
This basic organization scheme is derived from accepted approaches in the literature: adjacent blocks share a tag entry, yet they can only be co-allocated in a data entry if each of them compresses to at least a specific percentage of the cache line size.
Currently, only BDI~\cite{pekhimenko2012base}, C-Pack~\cite{chen2010c}, and FPCD~\cite{alameldeen2018opportunistic} are implemented, but the modularity of the compressors allows for simple implementation of other dictionary-based and pattern-based compression algorithms (e.g., only a few hours of development effort for a developer familiar with the code).

These replacement policies are a great example of gem5's modularity and how code developed for one purpose can be reused in many other parts of the simulator.
Current and future development is planned to increase the use of these flexible replacement policies.
For instance, we are planning to extend the TLB and other cache structures beyond the data caches to take advantage of the same replacement policies.
Additionally, although the aforementioned cache compression policies have only been applied to the classic caches, we are planning to use the same modular code to enable cache compression for the Ruby caches as well.

\subsection[Ruby Cache Model Improvements]{Ruby Cache Model Improvements}
\label{sec:ruby}

The Ruby cache model, originally from the GEMS simulator~\cite{MartinSBMXAMHW05}, is one of the key differentiating features of gem5.
The domain-specific language SLICC allows users to define new coherence protocols with high fidelity.
In mainline gem5, there are now 12 unique protocols including GPU-specific protocols, region-coherence protocols~\cite{PowerBasu2013-hsc}, research protocols like token coherence~\cite{MartinHill2003-tokenCoh}, and teaching protocols~\cite{NagarajanSorin2020-cohMCMPrimer}.

When gem5 was first released, Ruby had just been integrated into the project.
In the nine years since, Ruby and the SLICC protocols have become much more deeply integrated into the general gem5 memory system.
Today, Ruby shares the same replacement protocols (Section~\ref{sec:replacement}), the same port system to send requests into and out of the cache system, and the same flexible DRAM controller models (Section~\ref{sec:dramcontroller}).

Looking forward, we will be further unifying the Ruby and classic cache models.
Our goal is to one day have a unified cache model which has the composability and speed of the classic caches and the flexibility and fidelity of SLICC protocols.

\subsubsection[General Improvements]{General Improvements\footnote{by Nilay Vaish}}

Ruby now supports state checkpointing and restoration with warm cache.
This enables running simulations from regions of interest, rather than having to start fresh every time.
To enable checkpoints, we support accessing the memory system functionally i.e. without any notion of time or events.
The absence of timed events allows much higher simulation speeds.

Additionally, a new three level coherence protocol (\verb|MESI_Three_Level|) has been added to gem5.
For simplicity, this protocol was built on top of a prior two level protocol by adding an ``zero level'' (L0) cache at the CPU cores.
At the L0, the protocol has separate caches for instructions and data.
The L1 and the L2 caches are unified and do not distinguish between instructions and data.
The L0 and L1 caches are private to each CPU core while the L2 is shared across either all cores or a subset.

% \subsubsection[GPU Coherence Protocols]{GPU Coherence Protocols}

% Haven't heard anything, yet.

\subsubsection[Arm Support and Extensions]{Arm Support in Ruby Coherence Protocols\footnote{by Tiago M{\"u}ck}}

Until recently, configurations combining Ruby and multicore Arm systems were not properly supported.
We have revamped the \verb|MOESI_CMP_directory| protocol and made it the default when building gem5 for Arm.
Several issues that resulted in protocol deadlocks (especially when scaling up to many-core configurations) were fixed.
Other fixes include support for functional accesses, DMA bugs, and improved modeling of cache and directory latencies.
Additionally, support for load-locked/store-conditional (LL/SC) operations was added to the \verb|MESI_Three_Level| protocol, which enables it to be used with Arm as well.

\subsection[Garnet Network Model]{Garnet Network Model\footnote{By Srikant Bharadwaj and Tushar Krishna}}

The interconnection system within gem5 is modeled in various levels of detail and provides extensive
flexibility in terms of modeling modern systems.
The interconnect models are present within the cache-coherent ruby memory system of gem5.
It provides the ability to create arbitrary topologies – thereby constructing both homogeneous and heterogeneous systems.
There are two major variants of network models available within the ruby memory system today: simple and garnet.
The Simple network models the routers, links, and the latencies involved with minimal detailing.
This is great for simulations that can sacrifice interconnect detailing for faster simulation.
The Garnet model adds detailed router microarchitecture with cycle-level buffering, resource-contention and flow control mechanisms~\cite{}(N. Agarwal, T. Krishna, L. Peh and N. K. Jha, GARNET: A detailed on-chip network model inside a full-system simulator, 2009 IEEE International Symposium on Performance Analysis of Systems and Software, Boston, MA, 2009, pp. 33-42, doi: 10.1109/ISPASS.2009.4919636.).
This model is suitable for studies that focus on interconnection units and data flow patterns.

gem5 currently implements an upgraded Garnet 2.0 model which provides custom routing algorithms, routers/links that support heterogenous latencies, and standalone network simulation support.
These features allow detailed studies of on-chip networks as well as support for highly flexible topologies.
Garnet is moving to version 3.0 with the release of HeteroGarnet which is underway.
HeteroGarnet revamps Garnet to support the modern heterogenous systems such as 2.5D integration systems, MCM based architectures, and futuristic interconnect designs such as optical networks~\cite{}( S. Bharadwaj, J. Yin, B. Beckmann, T. Krishna, Kite: A Family of Heterogeneous Interposer Topologies Enabled via Accurate Interconnect Modeling, 2020 57th ACM/IEEE Design Automation Conference (DAC), San Francisco, CA, USA, 2020.).
We are also working to include support for recent work on routerless NoCs~\cite{}(\url{https://ieeexplore.ieee.org/abstract/document/8327032}, \url{https://ieeexplore.ieee.org/abstract/document/9065600}).

% New models
\subsection[GPU Compute Model]{GPU Compute Model\footnote{by Anthony Gutierrez}}
\label{sec:gpu}

GPUs have become an important part of the system design for high-performance computing, machine learning, and many other workloads.
Thus, we have integrated a compute-based GPU model into gem5~\cite{gutierrez-hpca-gpu} (\url{https://ieeexplore.ieee.org/document/8327041}).

\subsubsection[Autonomous Data-Race-Free GPU Tester]{Autonomous Data-Race-Free GPU Tester\footnote{by Tuan Ta}}

The Ruby coherence protocol tester is designed for CPU-like memory systems that implement relatively strong memory consistency models (e.g., TSO) and hardware-based coherence protocols (e.g., MESI).
In such systems, once a processor sends a request to memory, the request appears globally to the rest of the system.
Without knowing implementation details of target memory systems, the tester can rely on the issuing order of reads and writes to determine the current state of shared memory.
However, existing GPU memory systems are often based on weaker consistency models (e.g., sequential consistency for data-race-free) and implement software-directed cache coherence protocols (e.g., the VIPER Ruby protocol which requires explicit cache flushes and invalidations from software to maintain cache coherence).
The order in which reads and writes appear globally can be different from the order they are issued from GPU cores.
Therefore, the previous CPU-centric Ruby tester is not applicable to testing GPU memory systems.

The gem5 simulator currently supports an autonomous random data-race-free testing framework to validate GPU memory systems.
The tester works by randomly generating and injecting sequences of data-race-free reads and writes that are well synchronized by proper atomic operations and memory fences to a target memory system.
By maintaining the data-race freedom of all generated sequences, the tester is able to validate responses from the system under test.
The tester is also able to periodically check for forward progress of the system and report possible deadlock and livelock issues.
Once encountering a failure, the tester generates an event log that captures only memory transactions related to the failure, which significantly eases the debugging process.
Tuan Ta et al. showed how the tester effectively detected bugs in the implementation of VIPER protocol in gem5~\cite{Ta2019gputesting}.

\subsection[Runtime Power Modeling and DVFS Support]{Runtime Power Modeling and DVFS Support\footnote{by Stephan Diestelhorst}}
\label{sec:dvfs}

Virtually all processing today needs to consider not just aspects of performance, but also that of energy and power consumption.
Many systems are constrained by power or thermal conditions (mobile devices, boosting of desktop systems) or need to operate as energy efficiently as possible (in HPC and data centers).
We have added support to gem5 to model power-relevant silicon structures: voltage and frequency domains.
We have also added a model for enabling DVFS (dynamic voltage and frequency scaling) and support devices that allow for DVFS control by operating system governors and autonomous control.
Finally, we added an activity-based power modeling framework that measures key microarchitectural events, voltage, and frequency and allows detailed aggregation of power consumed over time similar to McPAT~\cite{LiAhn2009-mcpat, LiAhn2013-mcpat}.
Spiliopoulos et al. show that gem5's DVFS support can be integrated into both Linux and Android operating systems to provide end-to-end power and energy modeling~\cite{SpiliopoulosBHAK13}.
Additionally, these model have been extended to include power consumption caused by the activity of the SVE vector units.

\subsection[Timing-agnostic models: VirtIO and NoMali]{Timing-agnostic models: VirtIO and NoMali\footnote{By Andreas Sandberg}}

With the introduction of KVM support, it quickly became apparent that some of gem5’s device models, such as the IDE disk interface or the UART, were not efficient in a virtualized environment.
We also realized that these devices do not provide any relevant timing information in most experimental setups.
In fact, they are not even representative of the devices found in modern computer systems.
Similarly, when simulating mobile workload, such as Android, the GPU has a large impact on system behavior.
While it is possible to simulate an Android system without a GPU (the system resorts to software rendering), such simulations are wildly inaccurate for many CPU-side metrics~\cite{} (NoMali: Simulating a realistic graphics driver stack using a stub GPU).

These problems lead to the development of a new class of device timing-agnostic models in gem5.
For block devices, pass through file systems, and serial ports, we developed a set of VirtIO-based device models.
These models only provide limited memory system interactions and no timing. To solve the software rendering issue, we introduced a NoMali stub GPU model~\cite{} (NoMali: Simulating a realistic graphics driver stack using a stub GPU) that exposes the same register interface as an Arm Mali T-series and early G-series of GPUs.
This makes it possible to use a full production GPU driver stack in a simulated system without simulating the actual GPU.


\subsection[dist-gem5: Support for Distributed Computing]{dist-gem5: Support for Distributed Computing\footnote{by Gabor Dozsa}}

Haven't heard back, yet.

Note: I should probably reach out to Mohammad Alian.

Cite ``dist-gem5: Distributed Simulation of Computer Clusters'' and ``pd-gem5: Simulation Infrastructure for Parallel/Distributed Computer Systems''

\subsection[SystemC Integration]{SystemC Integration}

While the open and configurable architecture of gem5 is of particular interest
in academia, the industry's main tool for virtual prototyping is SystemC
Transaction Level Modelling (TLM)~\cite{systemc_ieee11}. Many hardware vendors
provide SystemC TLM models of their IP and there are tools, such as Synopsys
Platform Architect\footnote{\url{https://www.synopsys.com/verification/virtual-prototyping/platform-architect.html}},
that assist in building a virtual system and analyzing it. Also, many research
projects use SystemC TLM, as they benefit from the rich ecosystem of accurate
of-the-shelf models of real hardware components. However, there is a lack of
accurate and modifiable CPU models in SystemC since the model providers want to
protect their IP. This makes the combination of gem5 with SystemC very
attractive.

\subsubsection[gem5 to SystemC Bridge]{gem5 to SystemC Bridge\footnote{By Chistian Menard, Matthias Jung, Abdul Mutaal Ahmad, and Jeronimo Castrillon}}

SystemC TLM and gem5 were developed around the same time and are based on
similar underlying ideas. As a consequence, the hardware model used by TLM is
surprisingly close to the model of gem5. In both approaches, the system is
organized as a set of components that communicate by exchanging data packets
via a well defined protocol. The protocol abstracts over the physical
connection wires that would be used in a register transfer level (RTL)
simulation and thereby significantly increases simulation speed. In gem5,
components use \emph{master} and \emph{slave} ports to communicate to other
components, whereas in SystemC TLM, connections are established via
\emph{initiator} and \emph{target} sockets. Also, the three protocols
\emph{atomic}, \emph{timing} and \emph{functional} provided by gem5 find their
equivalent in the \emph{blocking}, \emph{non-blocking} and \emph{debug}
protocols of TLM. The major difference in both protocols is the treatment of
backpressure, which is implemented by a retry phase in gem5 and with the
exclusion rule of TLM.

\begin{figure}
    \centering
    \subcaptionbox{gem5 to SystemC\label{fig:example:gem5_to_sc}}{
      \includegraphics[height=4cm]{fig/gem5_to_systemc.pdf}
    }
    \subcaptionbox{SystemC to gem5\label{fig:example:sc_to_gem5}}{
      \includegraphics[height=4cm]{fig/systemc_to_gem5.pdf}
    }
    \subcaptionbox{both directions\label{fig:example:twoway}}{
      \includegraphics[height=4cm]{fig/twoway.pdf}
    }
    \caption{Possible scenarios for binding gem5 and SystemC.}
    \label{fig:gem5_tlm_example}
\end{figure}

The similarity of the two approaches enabled us to create a light-weight
compatibility layer. In our approach, co-simulation is achieved by hosting the
gem5 simulation on top of a SystemC simulation. For this, we replaced the gem5
discrete event kernel with a SystemC process that is managed by the SystemC
kernel. A set of transactors further enables communication between the two
simulation domains by translating between the two protocols as is shown in
Figure~\ref{fig:gem5_tlm_example}. This work was published
in~\cite{menard2017-system-systemc} where we documented our approach and showed
that the transaction between gem5 and TLM only introduces a low overhead of
about \(8\%\). The source code as well as basic usage examples can be found in
\texttt{util/tlm} of the gem5 repository.

\subsubsection[SystemC in gem5]{SystemC in gem5\footnote{By Gabriel Black}}

Alternatively, gem5 also has its own built in SystemC kernel and TLM implementation, and can run models natively as long as they are recompiled with gem5's SystemC header files.
These models can then use gem5's configuration mechanism and be controlled from Python, and, by using modified versions of the bridges developed to run gem5 within a SystemC simulation, TLM sockets can be connected to gem5's native ports.

This approach integrates models into gem5 more cleanly and fully since they are now first class gem5 models with access to all of gem5's APIs.
Existing models and \verb|c_main| implementations can generally be used as-is without any source level modifications; they just need to be recompiled against gem5's SystemC headers and linked into a gem5 binary.

While some parts of gem5's SystemC implementation are taken from the open source reference implementation (most of the data structure library and TLM), the core implementation is new and based off of the SystemC standard.
This means that code which depends on nonstandard features, behaviors, and implementation specific details of the reference implementation may not compile or work properly within gem5.
That said, gem5's SystemC kernel passes almost all of the reference implementation's test suite, with the exception of tests which are broken or explicitly test for implementation specific behavior or deprecated and undocumented features.

%\subsection[Arm Fastmodel]{Arm Fastmodel\footnote{By Gabriel Black}}

Haven't heard anything back, yet.

% \subsection[gem5 and SST Integration]{gem5 and SST Integration\footnote{by Curtis Dunham}}

Haven't head back, yet.

% Infrastructure
\subsection[Syscall Emulation Improvements]{Syscall Emulation Improvements\footnote{by Brandon Potter}}

Heard back, waiting for text.
This may be a bit late due to having to run it past the lawyers.

\subsection[Testing in gem5]{Testing in gem5\footnote{by Sean Wilson and Robert R. Bruce}}

Heard back, waiting for the text.

\subsection{Internal gem5 Improvements and Features}
\label{sec:internal}

It is important to recognize not only all of the ground-breaking additions to the models in gem5, but also general improvements to the simulation infrastructure.
Although these improvements do not always result in new research findings, they are a key \emph{enabling factor} for the research conducted using gem5.

The simulator core of gem5 provides support for event-driven execution, statistics, and many other important functions.
These parts of the simulator are some of the most stable components, and, as part of the gem5-20 release and in the subsequent releases, we will be defining stable APIs for these interfaces.
By making these interfaces \emph{stable} APIs, it will facilitate long-term support for integrating other simulators (e.g., SST~\ref{sec:sst} and SystemC~\ref{sec:systemc}) and projects that build off of gem5 (e.g., gem5-gpu~\cite{}, gem5-aladdin~\cite{}, and many others.)

\subsubsection[HDF5 Support]{HDF5 Support\footnote{by Andreas Sandberg}}

A major change in the latest gem5 release is the new statistics API.
While the driver for this API was to improve support for hierarchical statistics formats like HDF5~\cite{}, there are other more tangible benefits as well.
Unlike the old API where all statistics live in the same namespace, the new API introduces a notion of statistics groups.
In most typical use cases, statistics are bound to the current SimObject's group, which is then bound to its parent by the runtime.
This ensures that there is a tree of statistics groups that match the SimObject graph.
However, groups are not limited to SimObject.
Behind the scenes, this reduces the amount of boiler plate code when defining statistics and makes the code far less error prone.
The new API also brings benefits to simulation scripts.
A feature many users have requested in the past has been the ability to dump statistics for a subset of the object graph.
This is now possible by passing a SimObject to the stat dump call, which limits the statistics dump to that subtree of the graph.

With the new statistics API in place, it became possible to support hierarchical data formats like HDF5.
Unlike gem5's traditional text-based statistics files, HDF5 stores data in a binary file format that resembles a file system.
Unlike the traditional text files, HDF5 has a rich ecosystem of tools and official bindings for many popular languages, including Python and R.
 In addition to making analysis easier, the HDF5 backend is optimized for storing time series.
HDF5 files internally store data as N-dimensional matrices.
In gem5's implementation, we use one dimension for time and the remaining dimensions for the statistic we want to represent.
For example, a scalar statistic is represented as a 1-dimensional vector.
When analyzing such series using Python, the HDF5 backend imports such data sets as a standard NumPy array that can be used in common data analysis and visualization flows.
The additional data needed to support filesystem-like structures inside the stat files introduces some storage overheads.
However, these are quickly amortized when sampling statistics since the incremental storage needed for every sample is orders of magnitude smaller than the traditional text-based statistics format.

\subsubsection[Python 3]{Python 3\footnote{by Andreas Sandberg and Giacomo Travaglini}}

One of the main features which separates gem5 from other architectural simulators is its robust support for scripting.
The main interface to configuring and running gem5 simulations is Python scripts.
While the fundamental design has not changed, there have been many changes to the underlying implementation over the past years.
The original implementation frequently suffered from bugs in the code generated by SWIG and usability was hampered by poor adherence to modern standards in SWIG's C++ parser.
The move to PyBind11~\cite{} greatly improved the reliability of the bindings by removing the need for a separate C++ parser, and made it easier to expose new functionality to Python in a reliable and type-safe manner.

The move away from SWIG to PyBind11 provided a good starting point for the more ambitious project of making gem5 Python 3 compatible.
Making gem5 Python 3 compatible has not added any new features yet, but it ensures that the simulator will continue to run on Linux distributions that are released in 2020 and onwards.
It does however enable exciting improvements under the hood.
A couple of good examples are type annotations that can be used to enable better static code analysis and greatly improved string formatting.
Our ambition is to completely phase out Python 2 support in the near future to benefit from these new features.

\subsubsection[Asynchronous Modeling in gem5]{Asynchronous Modeling in gem5\footnote{by Giacomo Travaglini}}

The difficulties of writing a complex device/hw model within gem5 is that your model needs to be able to work and be representative of the simulated hardware in both atomic and timing mode.

For simple devices which only respond to requests, this is usually not a concern.
The situation gets worse when the device can send requests and response or has DMA capabilities.
A method generating and forwarding a read packet needs to differentiate between atomic and timing behavior by handling the first with a blocking operation (the read returns the value as soon as the forwarding method returns) and the second with a non-blocking call: the value will be returned later in time.
The situation becomes dramatic in timing mode if multiple sequential DMAs are stacked so that any read operation depends on previous ones; this is the case for page table walks for example.

This software design problem has been elegantly solved using coroutines.
Coroutines allow you to execute your task, checkpoint it, and resume it later from where you stopped.
To be more specific to our use case, you can tag your DMA packets with the coroutine itself, and you could resume the coroutine once the device receives the read response.

While waiting for coroutines to be fully supported in C++20, we've implemented a coroutine library within gem5 that allows developers to use coroutines to generate asynchronous models.
The coroutine class is built on top of a ``Fiber'' class, which was a pre-existing symmetric coroutine implementation, and it provides boost-like~\cite{} APIs to the user.

At the moment coroutines are used by the SMMUv3 model developed and the GICv3 ITS model (Interrupt Translation Service).
There are many other use cases for this API in other gem5 models, and we are planning on updating those models in the future.

\subsection[Updating Guest<->Simulator APIs]{Updating Guest$\leftrightarrow$Simulator APIs\footnote{By Gabriel Black}}
\label{sec:guest-sim}

It is sometimes helpful or necessary for gem5 to interact with the software running inside the simulation in some non-architectural way.
In Figure~\ref{fig:gem5-fs-fs}, the application under test may want to call a function \emph{in the gem5 simulator} or vice versa.
For instance, gem5 might want to intervene and adjust the guest's behavior to skip over some uninteresting function, like one that sets all of physical memory to zeroes, or which uses a loop to measure CPU speed or implement a delay.
It might also want to monitor guest behavior to know when something important like a kernel panic has happened.
Guest software might also want to purposefully request some behavior from gem5 such as requesting that gem5 exit, recording the current value of the simulation statistics, taking a checkpoint,  and reading or writing a file on the host, etc.

One way the simulator can react to guest behavior is by executing a callback when the guest executes a certain program counter (PC).
The PC would generally come from the table of symbols loaded with, for instance, an OS kernel, and would let gem5 detect when certain kernel functions were about to execute.
This mechanism has been improved to make it easier for different types of CPU models to implement.
These include the CPU models which use KVM and the ARM Fast Model based CPUs.

The gem5$\leftrightarrow$guest interaction might also be triggered by the application running on the guest itself.
One common way to use these mechanisms from within the guest is to use the ``m5'' utility which parses command line arguments and then triggers whatever gem5 behavior was requested.
This utility is in the process of being revamped so that support is consistent across ISAs, along with many other improvements including supporting all the back end mechanisms described above.

Because it is not possible to universally predict what PCs correspond to requests from the guest, a different signaling mechanism is necessary.
Traditional gem5 CPU models redefined unused opcodes from the target ISA for that purpose.
However, this mechanism is not universal.
For instance, when using the KVM-based CPU model instructions behave like they would on real hardware since they are running on real hardware.
In these special cases, we require other APIs.

Finally, the gem5 simulator code must be able to decipher the calling convention of guest code. Historically this was done in several different ways.
These were somewhat redundant, inconsistent, incomplete, and difficult to maintain.

We have implemented a new system of templates to pull apart a function's signature and marshal arguments from within the guest automatically.
Those arguments are then used to call an arbitrary function in gem5.
Once the function finishes, it can optionally return a value into the guest if it wants to override or just observe guest behavior.

For instance, suppose we had the function shown in Figure~\ref{fig:code1}.
If we wanted to call it from within the guest using calling convention AAPCS32, once gem5 had detected the call (as described above), it could call \verb|foo()| with arguments from the guest as shown in Figure~\ref{fig:code2}.

\begin{figure}
    \centering
    \begin{subfigure}{0.45\linewidth}
        \begin{lstlisting}[frame=single,basicstyle=\small]
int
foo(char bar, float baz)
{
    return (baz < 0) ? bar : bar + 1;
}
        \end{lstlisting}
        \caption{Example guest$\leftrightarrow$function.}
        \label{fig:code1}
    \end{subfigure}
    \hspace{2em}
    \begin{subfigure}{.38\linewidth}
        \begin{lstlisting}[frame=single,basicstyle=\small]
invokeSimcall<Aapcs32>(tc, foo);
        \end{lstlisting}
        \caption{Example gem5 code.}
        \label{fig:code2}
    \end{subfigure}
    \caption{Example use of new Guest$\leftrightarrow$Simulator APIs}
\end{figure}




% \section{Other work building off of gem5}

% Note: Come up with a better name.

% Accelerated simulated fault injection testing - \url{https://ieeexplore.ieee.org/document/8109288/}

% A Framework for Non-intrusive Trace-driven Simulation of Manycore Architectures with Dynamic Tracing Configuration - \url{https://link.springer.com/chapter/10.1007/978-3-030-03769-7_28}

% gem5-gpu Power et al.

% gem5-aladdin Shao et al.

% Many others.

% Note: May want to add something about \url{https://community.arm.com/developer/ip-products/system/b/soc-design-blog/posts/simplifying-workload-modelling-with-amba-atp-engine}


\section{Conclusion}

Over the past nine years, the gem5 simulator has become an increasingly important tool in the computer architecture research community.
This paper describes the significant strides taken to improve this community-developed infrastructure.
Looking forward, with the continued support of the broader computer architecture research community, the gem5 simulator will continue to mature and its use will continue to grow.
The community will continue to add new features, add new models, and increase the stability of the simulator.

The overarching goal of the future development of the gem5 simulator is to increase its user base by expanding its use both within the computer architecture community and in other computer systems research fields.
To accomplish this goal, we will be providing ``known-good'' configurations and other tools to enable reproducible computer system simulation.
We will also provide more user support to broaden the gem5 community through improved documentation and learning materials.
Through these efforts, we look forward to continue to grow and improve the gem5 simulation infrastructure through the next 20 years of computer system development.

\section{Acknowledgements}
\label{sec:acks}

The development of gem5 is community-driven and distributed.
The contributions to the gem5 community go beyond just the source code, and many people who have contributed to the broader gem5 community are not acknowledged here.

We would like to specially acknowledge the late Nathan Binkert.
Nate was a driving force behind the creation of gem5 and without his vision and his dedication to code quality this open-source community infrastructure would not be the success that it is today.

The gem5 project management committee consists of Bradford Beckmann, Gabriel Black, Anthony Gutierrez, Jason Lowe-Power (chair), Steven Reinhardt, Ali Saidi, Andreas Sandberg, Matthew Sinclair, Giacomo Travaglini, and David Wood.
Previous members include Nathan Binkert, and Andreas Hansson.
The project management committee manages the administration of the project and ensures that the gem5 community runs smoothly.

This work is supported in part by the National Science Foundation (CNS-1925724, CNS-1925485, CNS-1850566, and many others) and Brookhaven National Laboratory.
Google has donated resources to host gem5's codes, code review, continuous integration, and other web-based resources.

This work was partially completed with funding from the European Union's Horizon 2020 research and innovation programme under project Mont-Blanc 2020, grant agreement 779877.

We would also like to thank all of the other contributors to gem5 including
Chris Adeniyi-Jones, Michael Adler, Neha Agarwal, John Alsop, Lluc Alvarez, Ricardo Alves, Matteo Andreozzi, Nils Asmussen, Ruben Ayrapetyan, Erfan Azarkhish, Akash Bagdia, Jairo Balart, Marco Balboni, Marc Mari Barcelo, Andrew Bardsley, Isaac S\'anchez Barrera, Maurice Becker, Brad Beckmann, Rizwana Begum, Glenn Bergmans, Umesh Bhaskar, Nathan Binkert, Sascha Bischoff, Geoffrey Blake, Maximilien Breughe, Kevin Brodsky, Ruslan Bukin, Pau Cabre, Javier Cano-Cano, Emilio Castillo, Jiajie Chen, James Clarkson, Stan Czerniawski, Stanislaw Czerniawski, Sandipan Das, Nayan Deshmukh, Cagdas Dirik, Xiangyu Dong, Gabor Dozsa, Ron Dreslinski, Curtis Dunham, Alexandru Dutu, Yasuko Eckert, Sherif Elhabbal, Hussein Elnawawy, Marco Elver, Chris Emmons, Fernando Endo, Peter Enns, Matt Evans, Mbou Eyole, Marjan Fariborz, Matteo M. Fusi, Giacomo Gabrielli, Santi Galan, Victor Garcia, Mrinmoy Ghosh, Pritha Ghoshal, Riken Gohil, Rekai Gonzalez-Alberquilla, Brian Grayson, Samuel Grayson, Edmund Grimley-Evans, Thomas Grocutt, Joe Gross, David Guillen-Fandos, Deyaun Guo, Tony Gutierrez, Anders Handler, David Hashe, Mitch Hayenga, Blake Hechtman, Javier Bueno Hedo, Eric Van Hensbergen, Joel Hestness, Mark Hildebrand, Matthias Hille, Rune Holm, Chun-Chen Hsu, Lisa Hsu, Hsuan Hsu, Stian Hvatum, Tom Jablin, Nuwan Jayasena, Min Kyu Jeong, Ola Jeppsson, Jakub Jermar, Sudhanshu Jha, Ian Jiang, Dylan Johnson, Daniel Johnson, Rene de Jong, John Kalamatianos, Khalique, Do\u{g}ukan Korkmazt\"urk, Georg Kotheimer, Djordje Kovacevic, Robert Kovacsics, Rohit Kurup, Anouk Van Laer, Jan-Peter Larsson, Michael LeBeane, Jui-min Lee, Michael Levenhagen, Weiping Liao, Pin-Yen Lin, Nicholas Lindsay, Yifei Liu, Gabe Loh, Andrew Lukefahr, Palle Lyckegaard, Jiuyue Ma, Xiaoyu Ma, Andriani Mappoura, Jose Marinho, Bertrand Marquis, Maxime Martinasso, Sean McGoogan, Mingyuan, Monir Mozumder, Malek Musleh, Earl Ou, Xin Ouyang, Rutuja Oza, Andrea Pellegrini, Arthur Perais, Adrien Pesle, Polydoros Petrakis, Anis Peysieux, Christoph Pfister, Sujay Phadke, Ivan Pizarro, Matthew Poremba, Brandon Potter, Siddhesh Poyarekar, Nathanael Premillieu, Sooraj Puthoor, Jing Qu, Prakash Ramrakhyani, Steve Reinhardt, Isaac Richter, Paul Rosenfeld, Shawn Rosti, Ali Saidi, Karthik Sangaiah, Ciro Santilli, Robert Scheffel, Sophiane Senni, Korey Sewell, Faissal Sleiman, Maximilian Stein, Po-Hao Su, Chander Sudanthi, Kanishk Sugand, Dam Sunwoo, Koan-Sin Tan, Michiel Van Tol, Erik Tomusk, Christopher Torng, Ashkan Tousi, Sergei Trofimov, Avishai Tvila, Ani Udipi, Jordi Vaquero, Llu\'is Vilanova, Ilias Vougioukas, Wade Walker, Yu-hsin Wang, Bradley Wang, William Wang, Moyang Wang, Zicong Wang, Vince Weaver, Uri Wiener, Sean Wilson, Severin Wischmann, Willy Wolff, Yuan Yao, Jieming Yin, Bjoern A. Zeeb, Dongxue Zhang, Tao Zhang, Xianwei Zhang, Zhang Zheng, Chuan Zhu, and Chen Zou.

\bibliographystyle{ACM-Reference-Format}
\bibliography{references}

\end{document}
